% LaTeX UTILITIES ==============================================================

% Left/Right -------------------------------------------------------------------
% These two-letter command names mostly come from old personal conventions.
\newcommand{\pr}[1]{\left( #1 \right)} % PaRentheses
\newcommand{\sq}[1]{\left[ #1 \right]} % SQuare brackets
\newcommand{\cb}[1]{\left\lbrace #1 \right\rbrace} % Curly Braces
\newcommand{\vv}[1]{\left\lvert #1 \right\rvert} % single pipes (vert vert)
\newcommand{\VV}[1]{\left\lVert #1 \right\rVert} % double pipes (Vert Vert)
\newcommand{\ag}[1]{\left\langle #1 \right\rangle} % AnGle brackets
\newcommand{\fl}[1]{\left\lfloor #1 \right\rfloor} % FLoor
\newcommand{\cl}[1]{\left\lceil #1 \right\rceil} % CeiLing

% Left scripts -----------------------------------------------------------------
\newcommand{\lsub}[2]{{\vphantom{ #1 }}_{ #2 } #1}
\newcommand{\lsup}[2]{{\vphantom{ #1 }}^{ #2 } #1}
\newcommand{\lsubp}[2]{{\vphantom{ #1 }}_{ #2 }^{ #3 } #1}

% Functions --------------------------------------------------------------------
\newcommand{\par}[1]{\mathopen{} \pr{ #1 } \mathclose{}} % PARameter
\newcommand{\sqpar}[1]{\mathopen{} \sq { #1 } \mathclose{}}
\newcommand{\func}[2][]{\operatorname{ #2 }_{ #1 }\,}
\newcommand{\parfunc}[2][]{\operatorname{ #2 }_{ #1 }\par}
\newcommand{\sqparfunc}[2][]{\operatorname{ #2 }\_{ #1 }\sqpar}

% Aligned ----------------------------------------------------------------------
\newcommand{\align}[1]{\begin{aligned} #1 \end{aligned}}

% Cases ------------------------------------------------------------------------
\newcommand{\cases}[1]{\begin{cases} #1 \end{cases}}
\newcommand{\if}[1]{\text{if \( #1 \)}}
\newcommand{\otherwise}{\text{otherwise}}
\newcommand{\ow}{\otherwise}
\newcommand{\since}[1]{\text{since \( #1 \)}}

% Slash Fraction ---------------------------------------------------------------
\newcommand{\flac}[2]{\left. #1 \middle/ #2 \right.}

% GENERAL UTILITIES ============================================================

% Left/Right aliases -----------------------------------------------------------
\newcommand{\tuple}{\pr}
\newcommand{\array}{\sq}
\newcommand{\arr}{\sq}
\newcommand{\set}{\cb}
\newcommand{\abs}{\vv}
\newcommand{\size}{\vv}
\newcommand{\modulus}{\vv}
\newcommand{\norm}{\VV}
\newcommand{\length}{\VV}
\newcommand{\floor}{\fl}
\newcommand{\ceiling}{\cl}
\newcommand{\ceil}{\cl}

% Miscellaneous ----------------------------------------------------------------
\newcommand{\defequals}{\mathrel{\mathop:}=}
\newcommand{\defeq}{\defequals}
\newcommand{\deq}{\defequals}
\newcommand{\pair}[2]{\tuple{ #1 , #2 }}
\newcommand{\range}[2]{{ #1 }\,{..}\,{ #2 }}
\newcommand{\index}[1]{\mathopen{} \sq{ #1 } \mathclose{}}
\newcommand{\idx}{\index}
\newcommand{\divides}{\mid}

% COMPLEX NUMBERS ==============================================================

% Miscellaneous ----------------------------------------------------------------
\newcommand{\conjugate}[1]{\overline{#1}}
\newcommand{\conj}{\conjugate}
\newcommand{\Realpart}[1][]{\parfunc[#1]{Re}}
\newcommand{\Re}{\Realpart}
\newcommand{\Imaginarypart}[1][]{\parfunc[#1]{Im}}
\newcommand{\Im}{\Imaginarypart}

% FLOATING POINT ===============================================================

% Miscellaneous ----------------------------------------------------------------
\newcommand{\float}{{fl}\par}
    % \newcommand{\fl}{\float} name conflict with floor
\newcommand{\repeat}[1]{\overline{ #1 }}
\newcommand{\rep}{\repeat}
\newcommand{\machineepsilon}{\epsilon\_{\mathit{mach}}}
\newcommand{\emach}{\machineepsilon}

% Operators --------------------------------------------------------------------
\newcommand{\fplus}{\oplus}
\newcommand{\fadd}{\fplus}
\newcommand{\fminus}{\ominus}
\newcommand{\fsub}{\fminus}
\newcommand{\ftimes}{\otimes}
\newcommand{\fmul}{\ftimes}
\newcommand{\fdivide}{\oslash}
\newcommand{\fdiv}{\fdivide}

% SETS =========================================================================

% Set building notation --------------------------------------------------------
\newcommand{\where}{\ \middle|\ }

% Common sets ------------------------------------------------------------------
\newcommand{\Naturals}{\mathbb{N}}
\newcommand{\Nats}{\Naturals}
\newcommand{\N}{\Naturals}
\newcommand{\Integers}{\mathbb{Z}}
\newcommand{\Ints}{\Integers}
\newcommand{\Z}{\Integers}
\newcommand{\Reals}{\mathbb{R}}
\newcommand{\R}{\Reals}
\newcommand{\Complex}{\mathbb{C}}
\newcommand{\C}{\Complex}
\newcommand{\Polynomials}[2]{P*{ #1 }\par{ #2 }}
\newcommand{\Nomials}{\Polynomials}
\newcommand{\Matrices}[3]{M*{ #1 , #2 }\par{ #3 }}
\newcommand{\Mats}{\Matrices}

% Intervals --------------------------------------------------------------------
\newcommand{\cc}[2]{\left[ #1 , #2 \right]} % Closed-Closed
\newcommand{\co}[2]{\left[ #1 , #2 \right)} % Closed-Open
\newcommand{\oc}[2]{\left( #1 , #2 \right]} % Open-Closed
\newcommand{\oo}[2]{\left( #1 , #2 \right)} % Open-Open

% Set operations ---------------------------------------------------------------
\newcommand{\union}{\cup}
\newcommand{\Union}{\bigcup}
\newcommand{\intersect}{\cap}
\newcommand{\Intersect}{\bigcap}
\newcommand{\directsum}{\oplus}
\newcommand{\dsum}{\directsum}

% LINEAR ALGEBRA ===============================================================

% Bases ------------------------------------------------------------------------
\newcommand{\Span}[1][]{\func[#1]{Span}}
\newcommand{\rank}[1][]{\parfunc[#1]{rank}}
\newcommand{\dim}[1][]{\parfunc[#1]{dim}}

% Matrices ---------------------------------------------------------------------
\newcommand{\matrix}[1]{{\begin{bmatrix} #1 \end{bmatrix}}}
\newcommand{\mat}{\matrix}
\newcommand{\vector}{\matrix}
\newcommand{\vect}{\vector}
\newcommand{\diagonal}[1][]{\parfunc[#1]{diag}}
\newcommand{\diag}{\diagonal}
\newcommand{\trace}[1][]{\parfunc[#1]{tr}}
\newcommand{\tr}{\trace}
\newcommand{\inverse}[1]{#1^{-1}}
\newcommand{\inv}{\inverse}
\newcommand{\transpose}[1]{#1^{T}}
\newcommand{\trans}{\transpose}
\newcommand{\conjtrans}[1]{#1^{\*}}
\newcommand{\conjt}{\conjtrans}
\newcommand{\ctrans}{\conjtrans}
\newcommand{\ct}{\conjtrans}

% Matrix subspaces -------------------------------------------------------------
\newcommand{\Columnspace}[1][]{\parfunc[#1]{Col}}
\newcommand{\Colspace}{\Columnspace}
\newcommand{\Col}{\Columnspace}
\newcommand{\Rowspace}[1][]{\parfunc[#1]{Row}}
\newcommand{\Row}{\Rowspace}
\newcommand{\Nullspace}[1][]{\parfunc[#1]{Null}}
\newcommand{\Null}{\Nullspace}
\newcommand{\LNullspace}[2][]{\Nullspace[#1]{\trans{ #2 }}}
\newcommand{\LNull}{\LNullspace}

% Linear mappings --------------------------------------------------------------
\newcommand{\Range}[1][]{\parfunc[#1]{Range}}
\newcommand{\Kernel}[1][]{\parfunc[#1]{Ker}}
\newcommand{\Ker}{\Kernel}
\newcommand{\nullity}[1][]{\parfunc[#1]{nullity}}

% Coordinates ------------------------------------------------------------------
\newcommand{\coordinates}[2]{\matrix{ #1 }_{ #2 }}
\newcommand{\coords}{\coordinates}
\newcommand{\changecoordinates}[2]{\lsub{P}{ #1 }_{ #2 }}
\newcommand{\coc}{\changecoordinates}
% matrix of #2 w.r.t. bases #3 and #1
\newcommand{\mapmatrix}[3]{\lsub{\matrix{ #2 }}{ #1 }_{ #3 }}
\newcommand{\mapmat}{\mapmatrix}
\newcommand{\bmatrix}[2]{{\mat{ #1 }_{ #2 }}}
\newcommand{\bmat}{\bmatrix}

% Inner products ---------------------------------------------------------------
\newcommand{\innerproduct}[2]{\ag{ #1 , #2 }}
\newcommand{\innerprod}{\innerproduct}
\newcommand{\iprod}{\innerproduct}
\newcommand{\projection}[1]{\parfunc[#1]{proj}}
\newcommand{\proj}{\projection}
\newcommand{\perpendicular}[1]{\parfunc[#1]{perp}}
\newcommand{\perp}{\perpendicular}
\newcommand{\orthogonalcomplement}[1]{{ #1 }^{\bot}}
\newcommand{\orthocomplement}{\orthogonalcomplement}
\newcommand{\orthocomp}{\orthogonalcomplement}
\newcommand{\ocomp}{\orthogonalcomplement}

% Common linear algebra variables ----------------------------------------------
\newcommand{\B}{\mathcal{B}}
\newcommand{\U}{\mathbb{U}}
\newcommand{\V}{\mathbb{V}}
\newcommand{\W}{\mathbb{W}}
\newcommand{\a}{\vec{a}}
\newcommand{\b}{\vec{b}}
\newcommand{\e}{\vec{e}}
\newcommand{\u}{\vec{u}}
\newcommand{\v}{\vec{v}}
\newcommand{\w}{\vec{w}}
\newcommand{\x}{\vec{x}}
\newcommand{\y}{\vec{y}}
\newcommand{\z}{\vec{z}}
\newcommand{\zero}{\vec{0}}

% CALCULUS =====================================================================

% Derivative notations
\newcommand{\dd}[3][]{\frac{d^{#1} {#2}}{d {#3}^{#1}}}
\newcommand{\pdd}[3][]{\frac{\partial^{#1} {#2}}{\partial {#3}^{#1}}}
\newcommand{\prm}{^\prime}
\newcommand{\pprm}{^{\prime\prime}}

% Logarithms
\newcommand{\lnp}{\ln\par}
\newcommand{\logp}[1][]{\log\_{#1}\par}

% Trigonometry
\newcommand{\cosp}[1][]{\cos^{#1}\par}
\newcommand{\sinp}[1][]{\sin^{#1}\par}
\newcommand{\tanp}[1][]{\tan^{#1}\par}
\newcommand{\secp}[1][]{\sec^{#1}\par}
\newcommand{\cscp}[1][]{\csc^{#1}\par}
\newcommand{\cotp}[1][]{\cot^{#1}\par}

% GRAPHS =======================================================================

% Notation
\newcommand{\edge}{\pair}
\newcommand{\graph}{\pair}

% Vertices
\newcommand{\degree}[1][]{\parfunc[#1]{deg}}
\newcommand{\deg}{\degree}
\newcommand{\indegree}[1][]{\parfunc[#1]{indegree}}
\newcommand{\indeg}{\indegree}
\newcommand{\ideg}{\indegree}
\newcommand{\outdegree}[1][]{\parfunc[#1]{outdegree}}
\newcommand{\outdeg}{\outdegree}
\newcommand{\odeg}{\outdegree}

% Trees
\newcommand{\BFS}[1][]{\parfunc[#1]{BFS}}
\newcommand{\DFS}[1][]{\parfunc[#1]{DFS}}
\newcommand{\level}[1][]{\parfunc[#1]{level}}
\newcommand{\parent}[1][]{\parfunc[#1]{parent}}

% Weighted Graphs
\newcommand{\weight}{w\par}

% Paths
\newcommand{\distance}[2]{d\par{ #1 , #2 }}
\newcommand{\dist}{\distance}

% ALGORITHMS ===================================================================

% Big O Notation
\newcommand{\Oh}{O\par}
\newcommand{\oh}{o\par}
\newcommand{\Om}{\Omega\par}
\newcommand{\om}{\omega\par}
\newcommand{\Th}{\Theta\par}

% Decision Problems
\newcommand{\YES}{\text{YES}}
\newcommand{\NO}{\text{NO}}

% Complexity Classes
\newcommand{\reducesto}[1][]{\leq\_{#1}}
\newcommand{\P}{\text{P}}
\newcommand{\NP}{\text{NP}}
\newcommand{\complete}[1]{\text{\(#1\)-complete}}
\newcommand{\SAT}[1]{\text{\(#1\)-SAT}}
\newcommand{\HamPath}{\text{Ham-Path}}

% Numerical Methods
\newcommand{\LTE}{{LTE}}

% LOGIC ========================================================================

% Values
\newcommand{\true}{\text{true}}
\newcommand{\True}{\text{True}}
\newcommand{\TRUE}{\text{TRUE}}
\newcommand{\false}{\text{false}}
\newcommand{\False}{\text{False}}
\newcommand{\FALSE}{\text{FALSE}}

% Operators
\newcommand{\AND}{\and}
\newcommand{\OR}{\or}
\newcommand{\NOT}{\neg}

% RELATIONAL DATABASES =========================================================

% Syntax
\newcommand{\attribute}[1]{\textit{#1}}
\newcommand{\attr}{\attribute}
\newcommand{\relation}[1]{\texttt{#1}}
\newcommand{\reln}{\relation}
\newcommand{\string}[1]{\text{“#1”}}
\newcommand{\str}{\string}

% Core Operators
\newcommand{\SELECTION}[1]{\sigma*{ #1 }\par}
\newcommand{\SELECT}{\SELECTION}
\newcommand{\PROJECTION}[1]{\pi*{ #1 }\par}
\newcommand{\PROJECT}{\PROJECTION}
\newcommand{\CROSSPRODUCT}{\times}
\newcommand{\CROSSPROD}{\CROSSPRODUCT}
\newcommand{\CROSS}{\CROSSPRODUCT}
\newcommand{\UNION}{\cup}
\newcommand{\DIFFERENCE}{-}
\newcommand{\DIFF}{\DIFFERENCE}
\newcommand{\RENAMING}[1]{\rho\_{ #1 }\par}
\newcommand{\RENAME}{\RENAMING}

% Derived Operators
\newcommand{\JOIN}[1]{\bowtie\_{ #1 }}
\newcommand{\NATURALJOIN}{\JOIN{}}
\newcommand{\INTERSECTION}{\cap}
\newcommand{\INTERSECT}{\INTERSECTION}

% STAT 333

\newcommand{\at}[2]{\left. { #2 } \right\rvert*{ #1 }}
\newcommand{\between}[3]{\left. { #3 } \right\rvert*{ #1 }^{ #2 }}

% Utilities
\newcommand{\FOLLOWS}{\sim}
\newcommand{\GIVEN}{\vert}
\newcommand{\cdf}[1][\,]{F*{#1}\!\par}
\newcommand{\pmf}[1][\,]{p*{#1}\!\par}
\newcommand{\pdf}[1][\,]{f*{#1}\!\par}
\newcommand{\mgf}[1][\,]{\phi*{#1}\!\par}
\newcommand{\dmgf}[2][]{\phi*{#1}^{\pr{#2}}\!\par}
\newcommand{\pgf}[1][\,]{\psi*{#1}\!\par}
\newcommand{\dpgf}[1][]{\psi\prm*{#1}\!\par}
\newcommand{\ddpgf}[1][]{\psi\pprm*{#1}\!\par}
\newcommand{\Var}{\parfunc{Var}}
\newcommand{\E}{\sqparfunc{E}}
\newcommand{\P}{\parfunc{P}}

% Discrete Distribution Names
\newcommand{\DU}[1][]{\parfunc[#1]{DU}}
\newcommand{\BIN}[1][]{\parfunc[#1]{BIN}}
\newcommand{\BERN}[1][]{\parfunc[#1]{BERN}}
\newcommand{\HG}[1][]{\parfunc[#1]{HG}}
\newcommand{\POI}[1][]{\parfunc[#1]{POI}}
\newcommand{\NB}[1][]{\parfunc[#1]{NB}}
\newcommand{\GEO}[1][]{\parfunc[#1]{GEO}}

% Continuous Distribution Names
\newcommand{\U}[1][]{\parfunc[#1]{U}}
\newcommand{\Beta}[1][]{\parfunc[#1]{Beta}}
\newcommand{\Erlang}[1][]{\parfunc[#1]{Erlang}}
\newcommand{\EXP}[1][]{\parfunc[#1]{EXP}}

% Discrete-Time Markov Chains
\newcommand{\prv}[1][]{\underline{\alpha}_{#1}}
\newcommand{\period}[1][]{d_{#1}\par}
\newcommand{\communicates}{\leftrightarrow}

\newcommand{\gcdp}[1][]{\gcd*{#1}\par}
\newcommand{\vp}[2]{f*{#1,#2}}
\newcommand{\fvp}[3]{\vp{#1}{#2}^{\pr{#3}}}
\newcommand{\tp}[2]{P\_{#1,#2}}
\newcommand{\tpn}[3]{\tp{#1}{#2}^{\pr{#3}}}

% STAT 330 Overrides

% \newcommand{\pmf}[1][\,]{f*{#1}\!\par}
% \newcommand{\mgf}[1][\,]{M*{#1}\!\par}
% \newcommand{\dmgf}[2][]{M\_{#1}^{\pr{#2}}\!\par}

% \newcommand{\Bernoulli}[1][]{\parfunc[#1]{Bernoulli}}
